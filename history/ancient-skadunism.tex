\textit{``But should the tribes of man fall into sin and anger the six greatest mountains too severely, all six will proclaim their fiery rage in tandem, and humanity will be removed from the world in their entirety.''} \footnote{Legate Harald III, \textit{A Canon of Skadun for the Masses}, 413 AC}\\


\index{Skadunism}
Of the world religions that emerged in ancient times few survive, and fewer still exert an equal or greater influence now than they did in the past. One of those faiths that survived the test of time is Skadunism (Clerical Altyrian: \textit{Ska\textipa{D\'u}n}), a family of animistic faiths identified by apocalypticism, a strong clergy presence, and a focus on penitence. Though sometimes considered a Toran Era faith, the roots of Skadunism reach back into the prehistory of Ullr.

\section{Origins in Mountainistic Animism}

Though true Skadunism had not begun to emerge until around the 16th century BC, the religion's roots stretch back millennia, into the prehistory of Ullr. Particularly in the North and East, Ullr was dominated by animistic and occasionally shamanistic belief systems, with polytheism being notably uncommon throughout the continent\footnote{A notable exception being [Voromje]}. What would eventually become a dominant trend in these early belief systems is a concept referred to as "Mountainisticism" by scholars.

\subsection{Mountainistic Animism in North Ullr}
\index{Mountainistic animism}

As defined by scholars of Ullr prehistory, "Mountainisticism" is a feature of animistic belief systems where every mountain is seen as possessing a powerful spirit. What sets mountainistic thought apart from conventional animism is the idea that mountain spirits are somehow more powerful or significant that those of anything else in the world. This mountainistic animism elevated mountains enough that some theologists describe the practice as deification, though the degree to which mountainistic animism glorified mountain spirits varied significantly with region.\par

Mountainistic animism was, logically, most prevalent in rugged, mountainous areas such as the majority of northeast Ullr. This belief system has not left any literary sources due to its age, but is attested to in cave murals and rock carvings. Collections of stone cairns\footnote{Deliberate man-made stacks of rocks, often serving as landmarks}, sometimes accompanied by bones or pottery shards, have also been unearthed all over north Ullr; some dating back into the neolithic. Some archaeologists believe that these were pan-tribal meeting sites for early mountainists. \par
\begin{quote}
``To me, it is hard to accept arguments that the cairn complexes of north Ullr -- such as Tzerach Steinhaufen -- are not a result of mountainistic faiths. Any one of my peers who would claim such a thing has clearly never set foot at one of these sites...[cairn complexes] are almost universally built in the shadow of mountains. To date, none have been found dating to before the year 2800 BC that do not have at least one mountain in sight. It is clear to me that cairn complexes are the earliest expression of Skadunist rural monumentalism, and for that reason must not be attributed to animistic faiths in general. I maintain that...no one will find sufficent evidence to the contrary.''\footnote{excerpt from archaeologist Vulkas Heinkel's presentation at the Pasaj Yokib international archaeology conference (984)}
\end{quote}

Mountainistic animism is also believed to be the source of volcanic apocalycpticism in Skadunism.\footnote{Skadunist apoctalypticism is described later in this chapter.} Easter Ullr is a volcanically active region, and many of the larger mountains are in reality stratovolcanoes, capable of erupting violently after long periods of dormancy. In all likelihood, mountains which had already become objects of worship then erupted, causing people to spread stories of the  destructive potential a mountain could bring. Combined with the personification and deification of mountains by mountanistic animists, and these eruptions could become associated with the rage and fury of powerful beings.

\subsubsection{Supervolcano Hypotheses}
\index{Skadunism!Supervolcano Hypothesis}

One hypothesis relating to mountainistic animism which has gained traction in recent years is the Supervolcano Hypothesis. Based on geological, historiographicical, and religious evidence, this hypothesis posits that the eruption of a supervolcano over four thousand years ago contributed to the formation of mountainistic and apocalyptic thought in Ullr. \par 

The evidence for this hypothesis comes in multiple parts. For one, there is a sediment layer with volcanic ash content found almost uniformly across Ullr and Vinayaka, dated to roughly the same period. This widespread contemporal ash fall is indicative of a large -- if not massive -- volcanic eruption, which is recent enough to have impacted human society and culture. Secondly, many cultures of Ullr and Vinayaka reference a `great darkening' near the beginning of creation.\footnote{[TODO: examples or quotes]} Advocates for the hypothesis claim this demonstrates multiple cultures remembering the obstruction of sunlight that would result from a massive ash plume entering the upper atmosphere. Lastly, Skadunist doctrine talks of a great eruption that wiped the world clean of its previous inhabitants before humans were brought to live upon it.\footnote{[TODO: examples or quotes]} This myth could be inspired by this historical eruption, modified by word of mouth over the millennia.\par 

Despite this evidence, the Supervolcano Hypothesis is still disputed. Some scholars maintain that human memory would not keep track of such an event in detail until it could enter Skadunist doctrine over 2200 years later. Others argue that since the hyothetical supervolcano's cauldera has not been found, the ash may be the result of many smaller eruptions within the span of a few years. One final source of opposition comes from fundamental Skadunist communities, which maintain that geology and vulcanism are heretical ideas introduced by non-Skadunists.\footnote{For fundamental Skadunists, scientific explanation of vulcanism as a phenomenon that affects individual areas and mountains undermines the Skadunist belief that any mountain can erupt.} Ultimately, whether or not the hypothesis is accurate only has a minor bearing on understanding Skadunist history and doctrine.

\subsection{Mountainistic Animism as Compared to Skadunism}


 


