	\section{Tinar Era (135 - 717 AC)}
		
		\subsection{Tinar Mubarat (cir. 20 - 196)}				
				The establishment and rise of the Bulkhai Sahinate began with the emergence of the Tinar tribe cir. 20, and ended in 196 with the establishment of the Bulkhai Sahinate. This period of history was marked by the decline of Valona, the decline and fall of the Firiik Mubarat Confederation, and the fall of the Lashnar Mubarat. The Tinar Mubarat expanded throughout the Arberesh region, managing to defeat and best its surrounding powers over its two-hundred year history: Valona in the Tinar-Valonan War of 119, Firiir in the Fourth Lashnar-Firiir War, and Lashnar following the Battle of Burk in 178. The Tinar Mubarat also managed to improve its relations with Valona following the Tinar unification of the peninsula after the Skudar Campaigns of 181-193, and managed to politically unify the Tinar Mubarat with Valona in 194. Following the Babuas' Coup of the Mubarat in 196, the Mubarat was transformed into the Bulkhai Sahinate, establishing the foundation for the regional superpower it would become.
				\subsubsection{Tinarlar I (cir.20 - 54 AC)}
					\textit{``Tinarlar I (11 BC - 54 AC) was the first leader of the Tinar Mubarat, who acted as the next major step in the spread of Dregunism across the Bulkhan Peninsula.''} \footnote{Mar Pionjin, \textit{History of the Dregunist Faith}, 671 AC}\\

						At the turn of the millenia, the Bulkhan Peninsula was a world in turmoil. The First Firiik-Lashnar War (15 - 17 AC) had ended disastrously for the Firiik Mubarat Confederation following the Battle of Uloshank, and saw the Lashnar Mubarat sweeping across the western Bulkhan coastline. The Confederation which had reigned supreme on the peninsula for almost five hundred years was beginning its hundred-year long decline. Following the loss, the Skudar and Burk Mubarats relinquished their hold on the Jurizid region, while the Allogron Mubarat began a process of removing its advisors and soldiers from the Arberesh region, beginning the removal of the Arberesh region from the Confederation's governing. Although the eastern Mubarats were confident in their ability to maintain their territories (unlike the western Mubarats, which were falling piecemeal to the rising Lashnar Mubarat), The Mubar of Allogron, Erudazir (29 BC - 31 AC), made the decision to release the Arberesh region from Firiik control. This loss of authority made it possible for the Fistins of the region to rise up and craft their own independent states, and began the almost century long period of warfare and bloodshed that would eventually be ended by the Tinar Mubarat's consolidation of Arberesh in 120 AC. 
						
						The Arberesh region, situated between the soutehrn Mubarat Confederation and the northern Valona Navalire, gave way to the unilateral rule of independent Fistin leaders (Fis) who ruled over small tracts of land. One of these Fis was named Tinarlar, now known as Tinarlar I and father of the Tinar Dynasty. The history of the Arberesh region has often been seen by curious historians as a difficult hurdle to jump over. During the three hundred years of nominal Firiik rule over the Arberesh region, few written records remain. Firiik influence over the region was light-handed, with the Confederation only demanding tribute every few years and intervening against Fistins growing too powerful if the need occured. Because of this, the tradition of documentation characteristic of the Firiik Mubarats, the later Tinar Mubarat, and the Bulkhai Sahinate is absent in the Arberesh region for much of its early history. As a result, little is known about Tinarlar's origins, his early life, or his lineage prior to the founding of the Tinar Fistin. Despite these difficulties, there are several facts that have been widely accepted among historians.
					
						Tinarlar himself claimed Firiik ancestry. Between the years 150 and 40 BC, the Firiik Mubarat Confederation had attempted to extend Firiik influence into the Arberesh region by sending Firiik "colonizers" to settle in the barbaric lands. The new Mubarats were initially successful, able to settle an expand four minor Mubarats north of the Mubarat River. However, continuous raids from the Arberesh against the new Mubarats finally proved too much to handle for the Confederation, which withdrew support for the northern Mubarats in 53 BC. The Firiik in the north formally dissolved them in the year 40 BC. Tinarlar claimed that he, through his mother's lineage, contained Firiik blood dating back to the last of the Firiik Mubarats north of the Mubarat River, Rumain, and claimed to be directly related to the last ruler of the Mubarat, Illintok II. Whether this claim was true is doubted among historians: Tinarlar would likely have lacked the ability to know whether he was related to the lost Mubarats due to the lack of familial records following the loss of the final Mubarats. However, what is factual is that he claimed Firiik ancestry, and was able to use this claim to exert authority over the surrounding, emerging Fistins. It would not be until the early 50s AC that this claim would truly come to help the emerging Tinar Fistin, when the Tinar Fistin gained the support of the minor Mubarat, Kilibir.
					
						Much of what is known about Tinarlar and the early Tinar Fistin comes from records written in the Kilibir Mubarat about Tinarlar's expansion and exploits. In the year 22 AC, Tinarlar I fought against the Fistin of Maliri at the Battle of Maliri, emerging victorious and establishing Tinar control over the Fistin. This victory was important for several reasons. Maliri was the holder of the holy Dregunist mountain of Malietr, and its position as the holder of Malietr had been backed by the Firiik Confederation for roughly 120 years. Tinarlar's victory over Malietr not only showed his prowess over stronger powers, but also highlighted a growing apathy in Firiik towards its northern provinces. The Confederation, at this time, seemed more focused on maintaining its current way of life and society than it was respecting the borders that it was responsible for upholding. Additionally, Tinar's conquest of the holy mountain of Malietr allowed Tinarlar to claim himself as the next ruler of the Dregunist faith. This claim was nominal more than anything else, as Tinar lacked the strength to back up this claim. Following Tinarlar's victory over Malietr, he turned east towards the Lower Mubarat River. Here, too, Tinarlar I won another victory against the Deggestin Fistin in 26 AC and extending his control further along the Mubarat River.
						
						Following Tinar's expansion and subjugation off the Deggesting and Malietr Fistins, Tinarlar slowed his conquests. Over the next roughly 30 years of his rule over Tinar, he would only go to war two more times. Tinarlar instead worked to consolidate his power over the subjugated Fistins, managing to bring the Fistins together and in line behind Tinar's will. Tinarlar also worked diligently to improve Tinar's position with the Kilibir Mubarat.
						
						Following the Firiik loss in the First Lashnar-Firiik War, many of the minor Mubarats found themselves placed in an awkward position. The Confederation, while defeated, was still a strong power. The remaining Mubarats - particularly the major Mubarats of Allogron, Kurfin, Durris, and Maidan - become closer and more intertwined. Together they demanded more payments and tributes from the minor Mubarats under their authority, citing the ineffectiveness of the minor Mubarats during the war against the upstart ex-slave empire and blaming the loss on them. The law was passed in Allogron following a unanimous declaration among the major Mubarats in the year 19 AC, upsetting several minor Mubarats. Additionally, following the Firiik loss there was a string of cancelled marriage plans between the major Mubarats and the minor. Several of the larger Mubarats - particularly Allogron and and Kurfin - felt it would be politically safer to increase the relations between themselves and other major Mubarats, and not the minor ones. This marks a continuing shift in thinking for the major Mubarats, which over the past century had increasingly seen the minor Mubarats as subjects rather than less-powerful equals. These movements angered many of the minor Mubarats and were the main reasons for the distancing of the Kilibir Mubarat away from the Confederation and towards new possibilities.
						
						The Kilibir Mubarat thus looked out to the north, where the disruption caused by the leaving of the Confederation opened up ample opportunities for a future for the Mubarat. Following Tinarlar's victory against Malietr, the Kilibir Mubar Firtik accepted Tinarlar's offer for diplomacy. Intrigued by the small Fistin's rapid gains and Tinarlar's claim to Firiik ancestry, Firtik saw in Tinarlar an opportunity for a politically safe partnership of sorts. It was Kilibir's support which allowed Tinarlar to rebound from his victory at Malietr and continue east against the Deggestin. While Kilibir continuously denied Tinarlar's offers for a unification through a marriage, the Mubarat nonetheless remained a vital ally for Tinar.
			\subsubsection{Clintir (54 - 87 AC)}
				\paragraph{}
					Tinarlar's death in 54 AC saw the succession of his eldest son, Clintir, to the title of Fis. Clintir seemed to follow in his father's footsteps of diplomacy and alliance-forging, favoring the use of his words over his sword during the earlier period of his reign. However, following the year 65 AC, pressure from his advisors and councillors would see him begin several campaigns to the north against the emerging Arberesh Fistins and Mubarats.
				
					Clintir's most important contribution to the success of the Tinar Fistin was his inclusion of foreign intellectuals and scholars. In particular, he opened Tinar's doors to Firiik refugees and emigrants. These Firiik scholars and councillors largely came through the Kilibir Mubarat in the late 50s, but by the 70s and 80s AC the Tinar Fistin was attracting Firiik emigrants from the Sixth Firiik Mubarat itself. Through these Firiik immigrants, the Tinar Fistin began to develop "proper" temples, writing systems and language, and learned from the immense archives of Firiik military history on how to better conduct its military affairs. Beginning in Clintir's reign, Tinar began to imitate Firiik architecture, writing, culture, and arts. The Dregunism of Firiir, while sparsely spread throughout the Arberesh region, became mandatory under the Tinar Fistin. It is through the Tinar Mubarat (and the subsequent Bulkhai Sahinate) that Dregunism would see itself become a staple of the peninsula. Work began on a Tinar central temple to Dregunism in the year 76, which would go on to become the immense Grand Temple as subsequent generations continued to build upon it. Finally, the Firiik language was made mandatory for all high-government positions during Clintir's final years in the mid 80s: partly due to continued Firiik imitation and partly due to an easier communication between the high-ranking Firiik immigrants and the documents they brought with them.

					Clintir's reign also focused on a major expansion of Tinar's territorial borders. To the south of the Mubarat River, Clintir cooperated with the Kilibir Mubarat to increase both Kilibir's and Tinar's holdings south of the Mubarat River. Tinar's holdings south of the Mubarat River were lined mainly along the river banks, as Clintir was wary of the giant Firiik Mubarat which, while in decline, still posed a powerful threat to the upstart Fistin should it wake. To the north, Clintir focused Tinar's full brutality. His northern campaigns did not begin until the year 69 AC, and began with the conquest of the Zid Fistin and a tentative alliance between Tinar and two other Fistins, Peirat and Alin. By the year 75 AC, Tinar had extended its territorial holdings along the Bulkhan Sea coastline, managing to reach the Suri River southern bank as well as extending inland to its first main rival, the Dunavidin Mubarat. The Dunavidin Mubarat controlled a large swath of land bordering the Valonan tributaries to the north and surpassing the strength of the Tinar Mubarat. Under Clintir, Tinar began continuous raids against its northern rival around the year 83, emerging successful on many and returning with greater prestige when victorious. His successes were numerous enough that several smaller Arberesh Fistins rallied to Tinar's flag by Clintir's death in 87, most notably the Laimejtjan. It is important to note this was mostly just a nominal cooperation against Dunavidin and not any formal alliance or subjugation.
			\subsubsection{Tinef (87 - 101 AC)}
				\paragraph{}
					Tinef succeeded his father in 87, coming to the head of an exceptionally large Fistin with several other Fistin's following his lead. His first two years were marred with complications when a man who claimed to be the bastard son of Clintir, Iskott, arrived to challenge Tinef's right to the Fis of Tinar. While Tinef was eventually successful, this early crisis was a cause for concern so early on in his reign. However, his victory over his supposed half-brother Iskott, restored the faith in him and allowed Tinef to lay the foundations for his own successful rule.
					
					Following Iskott's failed claim to power, the Fistin of Tinar had its hand in many regional affairs and was involved in numerous border conflicts with the Dunavidin Fistin. Tinef began following in his grandfather's and father's steps, continuing to create a complex web of marriages and alliances between Tinar, its subjects, and its friends. Tinef's main goal was to link Tinar and Kilibir by marriage so that Tinar could formally claim the title of Mubarat and the prestige and authority that came with such power. While the backing of Kilibir had been enough for Tinar to command respect from its neighbors, the support of a southern Mubarat meant less as Tinar's ambitions grew further and further away from it. In particular, the northern Arberesh holdings required more Tinar intervention than not to maintain order under Tinar's flag. His ambitions were useful and his actions would be a major reason for Tinar's continued success later on: in 93, Tinef's wife died of disease and he managed to secure a marriage for himself and a Kilibir high-ranking woman, doing the same with his son. This marriage and the birth of three children over the next four years would firmly establish Tinar as a Mubarat. Additionally, the expansive links Tinef was able to create would go on to expand the Tinarlar dynasty throughout much of the Arberesh region throughout the next century.
			
					During this time, the conflicts between Tinar and Dunavidin had trickled and almost stopped. While raids started by either were common, they rarely ventured further than the borders and never conclusively altered the balance of power between the two. Tinef's time focused on the south had also turned him blind to the advances of Dunavidin in the north, who by the turn of the century expanded its hold on the northern Arberesh Fistins and Mubarats. However, the Dunavidin faced their own problems at the end of the century. Fas Vabbila Dur, the Fis of Dunavidin, had recently established Dunavidin authority over two smaller Fistins, Zinda and Eliamer. Fas Vabbila Dur was almost eighty years old at the year 96, and divided the Dunavidin Fistin among his twin sons to avoid conflict following his death. He died soon after in the year 98, at which point his two sons were immediately at one another's throats for control of the entirety of the Dunavidin Fistin. One son, Fas Dur Moliki, asked for aid from Tinef and the Tinar Mubarat. Tinef wisely accepted to support Fas Dur Moliki against his twin brother, Fas Dur Moliko. The war lasted until the year 100, with the Tinar backed Fas Dur Moliki emerging victorious with full control over the Dunavidin lands. It was at this point that Tinef and his allies enacted their own covert plan, assassinating Fas Dur Moliki only three days after his official victory over his brother and installing Moliki's infant son as the ruler of Dunavidin. The Tinar Mubarat quickly executed Moliki's old advisors without even an Arberesh trial, accusing them of committing the assassination and conspiring against Dunavidin; Tinef then appointed advisors from Tinar to almost every position of important in Dunavidin. Seemingly overnight, Tinar's biggest rival had been turned into a quiet puppet state under Tinef's thumb.
			
					Tinef would die soon after, but his legacy would live on as the most eventful ruler of Tinar since Tinarlar I. It is during his reign that we see a significant uptick in Firiik documents regarding the Tinar Mubarat, especially following the union between Tinar and Kilibir and the official formation of the Tinar Mubarat in 94 AC. Following this time, the number of Firiik emigrants - particularly those from higher positions in the Firiir Mubarats - rose tremendously. So, too, did Firiik interest in Tinar. Many of the documents dated to this time period come from both Firiik scholars within and without Tinar. The Firiik Mubarat Confederation's interest leaned more towards worry of the emerging "strong-man" just north of their borders, which had managed to unify with a Firiik Mubarat and bring over half of the Arberesh region under its control. 

			\subsubsection{Gezuar I (101 - 135 AC)}
				\paragraph{}
					Gezuar I succeeded his father as the second true Mubar of Tinar. Similarly to his father, his early reign was marred with complications. Upon hearing of the death of the Mubar Tinef, the Firiik Mubarat Confederation brought together an army for a campaign against Tinar, hoping to crush them during the early years of Gezuar's succession. The new Mubar managed to use Firiik's overconfidence against them and led Tinar to several victories, culminating in major Battle of Halif towards the end of 102 AC. While Gezuar's military prowess aided Tinar to victory, the contribution of the Kilibir Mubarat cannot be understated. Kilibir's decision to side with Tinar over the Confederation that it was a part of marked another major stake in the Firiik Mubarat Confederation. The desertion of its members and allies as the Confederation declined would further drive the Confederation into the ground leading up to the Second, Third, and Fourth Lashnar-Firiik Wars.
					
					Tinar's union with the Kilibir Mubarat remained intact due to his marriage in 94 AC, which ensured continued cooperation between Tinar and Kilibir. Following the Tinar-Kilibir victory over the Firiik Mubarat Confederation, Kilibir left the Confederation and extended its control to the south by bringing several smaller Mubarats in line behind it. Evidence points to the Kilibir expansion worrying Tinar (hypocritically), however no major tensions rose due to this expansion as the "areas of importance" for Tinar and Kilibir differed: Kilibir wished to extend its influence to the south, over the members of the Firiik Mubarat Confederation, seeing the Arberesh Fistins and Mubarats as barbaric (save for Tinar); Tinar wished to exert control over the Fistins of the Arberesh region. 
					
					Tinar's dominance over the former Kilibir lands remained sure due to Gezuar's marriage to a Kilibir high lady, and so Tinar's continued power in the region south of the Mubarat River remained intact. His father's conquests to the north were a different story. While the lands of the former Duvadan had been subdued, many of the remaining tribes in the north began coaslescing into a single organization dedicating to stopping the continuous expansion of the Tinar Mubarat. They formed the League of Arber in 107, and began raids on Tinar's northern frontier lands. Gezuar understood that letting these raids happen with impunity would only hurt his and Tinar's claim on the northern lands of Arber, which were already tenuous at best.  Gezuar then launched a series of attacks on the northern tribes over the next ten years, with the intent being to terrorize and not to conquer. He would lead small armies into the homelands of several of these tribes, pillage and plunder and the lands setting fire to whatever they could, and then leaving with several hundred prisoners and slaves. His attacks in the northern tribes eventually had the intended effect, with the League of Arber dissolving in 114, but they also earned Gezuar the name "the Vile".
					
					Gezuar had other things to worry about than his name. In the north, the sleeping Valonan beast which for so long had ignored Tinar's expansions and authority in the region finally turned its attention to its southwest. Having Tinar raiders and warriors on Valona's southern borders became to omuch of a threat to Valona, particularly after the Valonan Council learned of an increased Firiir presence in the Mubarat. Under Gezuar's reign, not only Firiir scholars and intellectuals, but Firiir refugees, workers, and farmes all flocked north to the Mubarat. Firiir's continual losses at the hands of the Lashnar Mubarat caused many to begin the journey north, and the formal declaration of the Fourth Lashnar-Firiir War caused many on the frontier lands to flee north. This "Firiir Diaspora" was not immediate, but took place over the 100 years since the first Lashnar-Firiir War. While this movement of intellectuals north had been happening for decades, the time period between 95-160 marked the first major wave of immigration from the rural class of Firiir people.
					
					Because of this, Valonan Councillors were worried of Tinar's strength and its possible claim of the title of a new Firiir. Valona first tried in vain to reform the League of Arber with Valona at the head in 117, but Gezuar "the Vile"'s actions had frightened many into neutrality, as most of the tribal people which currently made up the lands had been alive during Gezuar's reign of terror in the region. Instead, Valona relied on mercenaries and men from its eastern holdings in the Dunavan Lowlands. In 118, Valona began preparations for a Tinar-Valonan war and began to secretly plan with one of the few remaining tribes willing to fight against Tinar, Bagyar. 
					
					Gezuar managed to get word of the Valonan preparations ahead of time and pre-emptively launched a devastating campaign into Valona in early 119. The coastal city of Dati was captured later that year, and the Valonan capital was under siege by early 120. While Tinar lacked the capacity to fully siege the gigantic city, the quick push north into the Valonan heartland had the desired effect and Gezuar forced Valona into a peace treaty which gave Tinar all Valonan lands south of the Purtukkal River. The victory over the giant of Valona not only gave Tinar firm legitimacy and power in the Arber region, but once and for all settled the question of the north's major power as Tinar had unseated even the sleeping giant of Valona. By Gezuar's death in 135, Tinar had complete control over the Arber region, either directly from its seat in in Tinar or indirectly through small tribal councils. The direct control of the Arber region would come in 162.
			\subsubsection{Zetnep (135 - 149 AC)}
				\paragraph{}
					The reign of Zetnep saw Tinar as, for the first time, a major power and force to be reckoned with on the Bulkhan Peninsula. Its victory over Valona in the Tinar-Valonan War of 119-120 saw Valona fall by the wayside in both prestige and authority, and the three great powers of the region became the Firiir Confederation, the Lashnar Mubarat, and the Tinar Mubarat. In 136, the tide of power on the peninsula began its next major change in the Fourth Lashnar-Firiir War where the Lashnar Mubarat launched what it hoped to be a final and decisive blow to the Firiir Confederation. Lashnar's declaration brought about uncertainty in Tinar: it was an obvious attempt to not only defeat Firiir, but to also conquer the ancient city and claim the title of the "next Firiir" for itself. Zetnep had decided that he wanted the title for his Mubarat, and did not believe that that Lashnar deserved it over his own Mubarat.
					
					Zetnep began preparations for a campaign against Firiir in 136 after hearing of the Lashnar invasion, and in late 137 felt prepared enough to launch his campaign. Sweeping through the unsuspecting northern Firiir towns, Zetnep did his best to curtail his soldiers from pillaging and vandalising the towns and peoples, hoping to win his claim to the lands both by the people and by his conquests. He arrived in Firiir and began his siege in early 138, which lasted only a week after a revolt from the inside overthrew the Firiir Mubar and the people of the city proclaimed Zetnep and Tinar the rulers of the city. Zetnep, now in legal control of Firiir's lands, felt obligated to defend his newly conquered lands from Lashnar, and marched west to meet Lashnar. They met at San, and emerged victorious after the day-long Battle of San. The Battle of San marked the first ever defeat of a Lashnar army by a foreign one of comparable size. Following his victory, Zetnep made the decision to not continue marching into Lashnar territories and instead decided to return home to begin rule of both Tinar and Firiir. However, his victory at the Battle of San also began a forty-year long series of conflicts between the Tinar Mubarat and Lashnar which would eventually end following a Tinar victory at the Battle of Burk and the dissolution of the Lashnar Mubarat and the division of its lands under the Tinar Mubarat.
			\subsubsection{Nurir (149 - 163 AC)}
				\paragraph{}
					The conflict between Lashnar and Tinar was tense following the Tinar capture of Firiir and the Lashnar defeat at the Battle of San. Nurir's main priority during his reign was not one of conquest, as had been his father's, but one of consoidation and decisive ruling. Tinar's hold on Firiir, while backed by the city of Firiir, was still tenuous at Zetnep's death despite his attempts to better incorporate Firiir. Nurir continued his father's efforts of consolidation by dividing Firiir into several different provinces, called Pulllums, and assigning Firiir born men to their positions. Nurir hoped that by giving Firiir autonomy under the Tinar rule, Nurir could better focus on suppressing revolts in the north and defending against Lashnar raids in the west. Nurir's strategy worked for a long time, although in the late 2nd century it led to a short-lived coalition and revolt against the Tinar Mubarat. 
					
					Nurir's rule remained relatively peaceful and free from any major insurrections, and even saw decisive victories over the Lashnar Mubarat resulting in gains in land following the Battles of Traktar, Fiktir, and Calinki. In 162, Nurir cemented direct Tinar rule over the remaining tribes of Arber which had been ruled indirectly for over three decades by the Mubarat.
			\subsubsection{Falir I (163 - 186 AC)}
				\paragraph{}
					Falir I was the Mubar of Tinar who led the final war against Lashnar, which resulted in the defeat of the Lashnar Mubarat and the final, undisputed rule of Tinar over the Bulkhan Peninsula. The Battle of Calinki in 162 was the last battle between Tinar and Lashnar led by Nurir, and it would not be until 178 that the armies of Tinar and Lashnar would meet again in such an official manner. Relations between the two Mubarats remained relatively peaceful following the Battle of Calinki, although both would order raids into the other's lands from time to time. 
					
					Despite their weaker position, the Lashnar Mubarat began conspiring with several tribal leaders of Skudar tribes to coordinate an attack on Tinar. The Lashnar Mubarat would lead an army into Tinar's Firiir holdings by marching along the southern Bulkhan coast, while the Skudar tribes would lead an invasion across the mountains into the Tinar heartland. Despite the disunity and disorganization of the Skudar tribes, Lashnar managed to bring several together by promising them wealth from the looting of Tinar. Following Falir's discovery of the plan against his Mubarat, he immediately planned for a counter-attack. Instead of marching across the flatlands of the Bulkhan Peninsula's southern coast, he marched his armied directly across the Mali Mountains and began an invasion of Lashnar's breadbasket, the Burk region. Here they began their Seige of Burk. The Lashnar Mubarat was able to mobilize its army meant for the south and rerouted it north to meet the invading Tinar forces, meeting at the Battle of Burk. The battle was a decisive Tinar victory, resulting in the complete annihalation of the Lashnar military. Tinar quickl sieged down Burk and made their way unimpedd to the Lashnar capital, where they forced a surrender and ended the forty years of continuous conflict between the two Mubarats.
					
					However, Falir's movements through Burk and northern Lashnar invited attacks from his foes in the north, and plunder-hungry tribal raiders from Skudar. Several towns in the northern regions of the Tinar Mubarat fell to revoling tribespeople, while three tribes from Skudar which had been prepared for an assault on Tinar with Lashnar's help decided to go ahead and invade anyway. They marched across the Mali Mountains, capturing the holy village of Mali and several towns which they promptly looted and sacked. Mali in particular was heavily damaged, with many priceless works of Firiik art and historical records being destroyed in the attack.
					
					Falir's revenge would come back on the Skudar tribesmen tenfold, as he vowed that his line would annihalate the tribes responsible for the attack on Mali and promised revenge on the revolters in the north. He held on to his promise, snuffing out the northern rebellion upon his return to Tinar. Although not living to see the end of them, Falir began the Skudar campaigns which lasted from 181-193 and saw the complete absorption of the Skudar tribes and highlands into the greater Tinar Mubarat.
			\subsubsection{Gezuar II (186 - 196 AC)}
				\paragraph{}
					Gezuar II came into power upon his father's death, which occured in a battle against Vradifost, a minor Skudar tribal leader. Gezuar famously rode to his father's corpse, took the helmet off of his father's head and replaced his own with it. Not wishing to dishearten those who had witnessed the death of the Mubar, he declared that the Mubar still lived - referring to himself, as he was now the Mubar - and launched his army forward to crush Vradifost and his forces.
					
					Gezuar reigned through a relatively prosperous period in the Mubarat's history. Disregarding the campaigns against the Skudar tribesmen, Tinar did not see itself embroiled in any major conflicts (especially conflicts which threatened its own sovereignty) for the first time since the Battle of San and the fall of the Lashnar Mubarat. Gezuar also spearheaded a movement to improve relations with Valona, so much so that there was even a proposition of political unity between Tinar and Valona.
					
					Valona had undergone several disastrous defeats at the hands of its revolting subjects in the east. Its loss of the Dunavan lowlands and the prosperous trade contacts that went with them left Valona a shell of its once great and expansive empire, and hemmorhaging money. Gezuar knew this, of course, and had been preparing to put an end to Valona undere his boot. But the Council of Valona instead came with the unprecedented offer of what amounted to a voluntary vassalization under Tinar: Valona would be freed of its debts to the Tokkan states as Tinar would pay them off; Valona would provide money and soldiers to the Tinar army, while Tinar would promise Valona protection.
					
					Gezuar accepted and Tinar and Valona unified in 193. However, Gezuar reneged on his promise and instead overthrew the Councilship of Valona in the Night of the Dancing Swords, a bloody coup which ended with Gezuar reforming Valona into a separate province of the Tinar Mubarat and placing his own man at the head of it, Jarraner. Several surviving members of the Night of the Dancing Swords vowed revenge on Gezuar's empty promise, and conspired with the religious order of Tinar, the Babuas, to themselves orchestrate a coup to replace the Mubar. The Babuas decided to overthrow Gezuar and form a Sahinate, a type of government where the political leaders are chosen by the religious order, which are themselves instructed by God to make their decision. 
					
					The coup against Gezuar occured in early 196. Although records of the time proclaim this date to be the first day of the year, it is highly unlikely that the coup occured on the said date and more likley that it occured in the fourth or fifth month of the year as the weather became colder. The Babuas were successful largely due to the backing of Gezuar's personal guard, which had become convinced that Gezuar was not chosen by God to be the leader of the Bulkhan people. The Babuas proclaimed that, now that the peninsula had become unified under a single entity, God was able to proclaim the Bulkhans as his chosen people and now had the ability to assign a leader of them through the Babuas. Despite Gezuar's anger, his hands were forced against action: to inflict violence upon the sacred order would damn him for all eternity and seal his fate amongst the people. Alas, that happened regardless, as many in Tinar believed the Babuas to be right and began the Revolt against Gezuar, which ended quickly as Gezuar's own guards turned against him. 
					
					Gezuar was given a choice: abdication or execution. Gezuar chose the latter, and was executed in early 196. The Babuas then set about reforming the Mubarat and choosing their own leader. They decided that, because Tinar had been the one to unify the peninsula, God would only allow someone from Tinar to rule the Bulkhans. So they proclaimed that God picked one of Gezuar's sons, Halur, as the new Sahin (the new title, meaning "Chosen") of the newly proclaimed Bulkhai. Many believe that the Babuas made the choice to promote Gezuar's son in order to stem any revolts that may have occured from those backing Gezuar as the Mubarat. Additionally, Halur was not the eldest but the fourth son of Gezuar, and had been sent to be raised in the Babua's temple. The Babuas likely decided upon Halur as the new Sahin because they could more easily manipulate him into doing their bidding, and thus they could maintain their control over the Sahinate indirectly. Although the role of the Babuas would diminish over the Sahinate's history, particularly following Enver's War of 360-361, they would mantain their ceremonial role in the Sahinate and would continue to be the selectors of the new Sahin until the Sahinate's dissolution in 717.
					
					Gezuar's death did not go over peacefully, and the next decade saw the new Bulkhai Sahinate embroiled in a major succession crisis which saw each of Gezuar's three sons declare their own Mubarat against their youngest brother, saw several Skudar tribes attempt to break free, and saw a minor Lashnar rebellion. The Babua's coup has been a long debated and poorly understood topic in Bulkhan history, as the reasons behind their coup, their overwhelming success in overthrowing the Mubarat and establishing the Sahinate, and their ability to maintain the Sahinate under their grip all have very few records and little documentation behind them.
					
					Regardless, the Babua's coup ushered in a new age for the peninsula. Following the Succession Wars, Halur would come out victorious against each of his three brothers and would go on to lead the Sahinate into its new chapter in history.
