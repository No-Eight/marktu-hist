\section{Firiik Era, Bulkhan Peninsula (1100 BC - 120 AC)}
	\subsection{Pre-Firiik Era (cir. 1500 - 1100 BC)}
	\subsection{Growth, Spread, and Influence (1100 - 562 BC)}
	\subsection{Firiik Mubarat Confederation (562 BC - 17 AC)}
	\subsection{Late Firiik Era, Decline (17 - 120 AC)}
\section{Tinar Era, Bulkhan Peninsula (120 - 717 AC)}
	\subsection{Tinar Mubarat (cir. 20 - 196 AC)}
		At the turn of the millenia, the Bulkhan Peninsula was a world in turmoil. The First Firiik-Lashnar War (15 - 17 AC) had ended disastrously for the Firiik Mubarat Confederation following the Battle of Uloshank, and saw the Lashnar Mubarat sweeping across the western Bulkhan coastline in the following decade. The Confederation, which had reigned supreme on the peninsula for almost five hundred years, began its hundred-year long decline. Following the loss, the Skudar and Burk Mubarats relinquished their hold on the Jurizid region, while the Allogron Mubarat began a process of removing its advisors and soldiers from the Arberesh region, beginning its removal from the Confederation's governing. Although the eastern Mubarats were confident in their ability to maintain their territories (unlike the western Mubarats, which were falling piecemeal to the rising Lashnar), The Mubar of Allogron, Erudazir (29 BC - 31 AC), made the decision to release the Arberesh region from Firiik control in 18 AC. This loss of authority made it possible for the Fistins of the region to rise up and craft their own independent states, and began the almost century long period of warfare and bloodshed that would eventually be ended by the Tinar Mubarat's consolidation of Arberesh in 120 AC. 
	\subsection{Early Bulkhai Sahinate (196 - 357 AC)}
	\subsection{Beliar-Reform Era, Bulkhai Sahinate (357 - 533 AC)}
	\subsection{Ximal-Reform Era, Bulkhai Sahinate (533 - 632 AC)}
	\subsection{Late Bulkhai Period (632 - 717 AC)}
\section{Feuding Cities Era, Bulkhan Peninsula (717 - 830 AC)}
	\subsection{Wars of Bulkhan Succession (717 - 729 AC)}
	\subsection{Post-Dissolution Period, Bulkhan Peninsula (729 - 796 AC)}
	\subsection{Pre-Colonial Period, Bulkhan Peninsula (796 - 830 AC)}
\section{Colonial Era, Bulkhan Peninsula (830 - 1020 AC)}

		
		
		
		