%\newgeometry{left=5cm, right=5cm, top=4cm}
\thispagestyle{plain}
	\section*{\begin{center}
			\Huge{Foreword}
	\end{center}}
		\lettrine[findent=2pt]{\fbox{\textbf{T}}}{ }o the astute reader, it should already be obvious that neither Yele'u University nor Maruk actually exist. For that matter, not one person or place mentioned in this book is to be found in the real world, unless something has changed very drastically since this book was penned. \textit{The History of Maruk: Overview and Chronology} is a detailed account of a fictional world's history, written by nostalgic moderators and players of the now-defunct roleplaying subreddit /r/CivWorldPowers. In these pages, we the authors endeavor to expand and elaborate on the events of a nationbuilding play-by-post we still remember fondly, while simultaneously improving on the original to achieve a more cohesive world and gripping narrative. As we do this, we also try our hardest to remain completely in character.\par 
		
		Once you have finished this brief preamble, the entire book is written to resemble an actual publication in the field of World History from the esteemed Yele'u University in Pylona. This work has, of course, been translated from its original [LUX Creole] to English for your enjoyment and ease of reading. We hope you find the peoples, places, cultures, and conflicts of Maruk to be as interesting as we do. After all, they gripped us enough that we sat down to write a book. Two, in fact.\par 
		
		As a final note to readers, \textit{The Historical Encyclopedia of Maruk} exists as a companion piece to \textit{The History of Maruk}. While this book provides a roughly chronological narrative of this world's unique history, the \textit{Encyclopedia} provides one with more focused and consolidated information on individual topics of interest. As this world is fabricated in its entirety, readers may find searching for information on people and places in the \textit{Encyclopedia} to be a fast and effective way of acquiring a deeper comprehension of the subject matter than this book can provide alone, so we recommend perusing both books in conjunction.\par
		
		And now, without further ado, we hope you enjoy a look into the detailed history of a world both very much--and yet in many ways nothing--like our own. \par
		
		\clearpage  
\restoregeometry
\tableofcontents
\newpage
\thispagestyle{plain}
\chapter*{Introduction}
\label{chap:intro}
\addcontentsline{toc}{chapter}{\nameref{chap:intro}}
	\lettrine[findent=2pt]{\fbox{\textbf{T}}}{ }\textit{he History of Maruk: Overview and Chronology} has one of the most storied pasts of any work still undergoing active editing and revision. This is no less than Yele'u University's overview of the entire world and its history, and has been in constant print for over a millennium. First penned as \textit{The Vinayaka Canon} in the year [], this work has become a living document, contributed to by scholars and historians the world over and curated by Yele'u University. This document is as much a compendium of the entire world's historical knowledge as it is a book, and is thus mandatory reading for all who would study at Yele'u University, particularly in the field of History.\par 
	This edition of \text{The History of Maruk} ends shortly after the tumultuous Millennium Wars, as more recent events are still so young as to be a subject of intense debate even within the walls of our institution. The document's scope therefore is everything from the beginning of knowable history to the Millennium Wars, though with careful attention paid to events after establishment of the first nation states and empires. What transpired before this point is only touched on briefly, as an Anthropology text will be better suited to breaking down the less knowable nature of history before such time.

\chapter*{The Traditional Opening}
\label{chap:tradopen}
\addcontentsline{toc}{chapter}{\nameref{chap:tradopen}}

\section{The Elul Orison}

\begin{center}\textit{Wom’in. K’in,Yol, K’am, K’ol. Xi’an, Dui’an, Li’an, Ye’an, Xu’an, Ka’an, Ge’an, Ku’an. Ha’um, Te’um, K’ak’um, Lu’um, K’en’um. Iik’um, Kep’um, Lus’um, Enhul’um.}\end{center} 

\bigskip

\lettrine[findent=2pt]{\fbox{\textbf{T}}}{ }he original version of this text was a compilation of Tetk’in \textit{Zamaru}. It began with a traditional chant of the 18 Holy Names, which are often collected into this Orison, known as the \textit{Elul Orison} and spoken at the beginning of endeavors for the Tetk'in. In the original version, this chant was immediately followed by the story of the Ages of the Tetk'in. It is only appropriate that we begin this more modern version of this text in similar fashion, for our paths before and behind us are built by the cobblestones of legends. 

\section{The Nine Ages of the Tetk'in}



\clearpage


