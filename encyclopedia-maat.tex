\begin{multicols}{2}
	\section{Ujiteru}
		The islands of Ujiteru\index{Ujiteru} is an archipelago off the western coast of Ma'at\index{Ma'at}. Fractitious since its early days, it is well known in history for being the seat where Sodai Kokenjin managed his empire.
		\subsection{Soda, Kokenjini}
					\index{Sodai Kokenjin} \index{Niruwatum}
			Sodai Kokenjin - sometimes known as Niruwatum, meaning 'Tiger of Light' - was a daimyo and later a shogun in Ujiteru and the founder of the Toran Empire\index{Ma'at!Toran Empire}. He was born in [BIRTH YEAR] before the Year of the Conclave, as a child of Daimyo [FATHER'S NAME] of the Sodai Clan\index{Ujiteru!Clans} and a noble woman of the [Yadayadayada denny take care of this pls I suck at this]. His childhood was spent in [place], where he learned the teachings of Aumwen\index{Aumwen}, the belief system which he would adhere to for the rest of his life and spread throughout his conquests.\par
			\subsubsection{Accession to the throne}
			\subsubsection{Unification of Ujiteru}
			\subsubsection{Conquest of the Trinity}
			\subsubsection{Conquest of Qian}
			\subsubsection{Conquest of Al-Eru}
			\subsubsection{Betrayal}
	\section{Tinar Era of Bulkhai}
		\subsection{Tinar Mubarat}
			The establishment and rise of the Bulkhai Sahinate began with the emergence of the Tinar tribe cir. 20, and ended in 196 with the establishment of the Bulkhai Sahinate. This period of history was marked by the decline of Valona, the decline and fall of the Firiik Mubarat Confederation, and the fall of the Lashnar Mubarat. The Tinar Mubarat expanded throughout the Arberesh region, managing to defeat and best its surrounding powers over its two-hundred year history: Valona in the Tinar-Valonan War of 119, Firiir in the Fourth Lashnar-Firiir War, and Lashnar following the Battle of Burk in 178. The Tinar Mubarat also managed to improve its relations with Valona following the Tinar unification of the peninsula after the Skudar Campaigns of 181-193, and managed to politically unify the Tinar Mubarat with Valona in 194. Following the Babuas' Coup of the Mubarat in 196, the Mubarat was transformed into the Bulkhai Sahinate, establishing the foundation for the regional superpower it would become.
			\subsubsection{Tinarlar I (cir.20 - 54 AC)}
					\textit{``Tinarlar I (11 BC - 54 AC) was the first leader of the Tinar Mubarat, who acted as the next major step in the spread of Dregunism across the Bulkhan Peninsula.''} \footnote{Mar Pionjin, \textit{History of the Dregunist Faith}, 671 AC}\\
						At the turn of the millenia, the Bulkhan Peninsula was a world in turmoil. The First Firiik-Lashnar War (15 - 17 AC) had ended disastrously for the Firiik Mubarat Confederation following the Battle of Uloshank, and saw the Lashnar Mubarat sweeping across the western Bulkhan coastline. The Confederation which had reigned supreme on the peninsula for almost five hundred years was beginning its hundred-year long decline. Following the loss, the Skudar and Burk Mubarats relinquished their hold on the Jurizid region, while the Allogron Mubarat began a process of removing its advisors and soldiers from the Arberesh region, beginning the removal of the Arberesh region from the Confederation's governing. Although the eastern Mubarats were confident in their ability to maintain their territories (unlike the western Mubarats, which were falling piecemeal to the rising Lashnar Mubarat), The Mubar of Allogron, Erudazir (29 BC - 31 AC), made the decision to release the Arberesh region from Firiik control. This loss of authority made it possible for the Fistins of the region to rise up and craft their own independent states, and began the almost century long period of warfare and bloodshed that would eventually be ended by the Tinar Mubarat's consolidation of Arberesh in 120 AC. 
						
						The Arberesh region, situated between the soutehrn Mubarat Confederation and the northern Valona Navalire, gave way to the unilateral rule of independent Fistin leaders (Fis) who ruled over small tracts of land. One of these Fis was named Tinarlar, now known as Tinarlar I and father of the Tinar Dynasty. The history of the Arberesh region has often been seen by curious historians as a difficult hurdle to jump over. During the three hundred years of nominal Firiik rule over the Arberesh region, few written records remain. Firiik influence over the region was light-handed, with the Confederation only demanding tribute every few years and intervening against Fistins growing too powerful if the need occured. Because of this, the tradition of documentation characteristic of the Firiik Mubarats, the later Tinar Mubarat, and the Bulkhai Sahinate is absent in the Arberesh region for much of its early history. As a result, little is known about Tinarlar's origins, his early life, or his lineage prior to the founding of the Tinar Fistin. Despite these difficulties, there are several facts that have been widely accepted among historians.
					
						Tinarlar himself claimed Firiik ancestry. Between the years 150 and 40 BC, the Firiik Mubarat Confederation had attempted to extend Firiik influence into the Arberesh region by sending Firiik "colonizers" to settle in the barbaric lands. The new Mubarats were initially successful, able to settle an expand four minor Mubarats north of the Mubarat River. However, continuous raids from the Arberesh against the new Mubarats finally proved too much to handle for the Confederation, which withdrew support for the northern Mubarats in 53 BC. The Firiik in the north formally dissolved them in the year 40 BC. Tinarlar claimed that he, through his mother's lineage, contained Firiik blood dating back to the last of the Firiik Mubarats north of the Mubarat River, Rumain, and claimed to be directly related to the last ruler of the Mubarat, Illintok II. Whether this claim was true is doubted among historians: Tinarlar would likely have lacked the ability to know whether he was related to the lost Mubarats due to the lack of familial records following the loss of the final Mubarats. However, what is factual is that he claimed Firiik ancestry, and was able to use this claim to exert authority over the surrounding, emerging Fistins. It would not be until the early 50s AC that this claim would truly come to help the emerging Tinar Fistin, when the Tinar Fistin gained the support of the minor Mubarat, Kilibir.
					
						Much of what is known about Tinarlar and the early Tinar Fistin comes from records written in the Kilibir Mubarat about Tinarlar's expansion and exploits. In the year 22 AC, Tinarlar I fought against the Fistin of Maliri at the Battle of Maliri, emerging victorious and establishing Tinar control over the Fistin. This victory was important for several reasons. Maliri was the holder of the holy Dregunist mountain of Malietr, and its position as the holder of Malietr had been backed by the Firiik Confederation for roughly 120 years. Tinarlar's victory over Malietr not only showed his prowess over stronger powers, but also highlighted a growing apathy in Firiik towards its northern provinces. The Confederation, at this time, seemed more focused on maintaining its current way of life and society than it was respecting the borders that it was responsible for upholding. Additionally, Tinar's conquest of the holy mountain of Malietr allowed Tinarlar to claim himself as the next ruler of the Dregunist faith. This claim was nominal more than anything else, as Tinar lacked the strength to back up this claim. Following Tinarlar's victory over Malietr, he turned east towards the Lower Mubarat River. Here, too, Tinarlar I won another victory against the Deggestin Fistin in 26 AC and extending his control further along the Mubarat River.
						
						Following Tinar's expansion and subjugation off the Deggesting and Malietr Fistins, Tinarlar slowed his conquests. Over the next roughly 30 years of his rule over Tinar, he would only go to war two more times. Tinarlar instead worked to consolidate his power over the subjugated Fistins, managing to bring the Fistins together and in line behind Tinar's will. Tinarlar also worked diligently to improve Tinar's position with the Kilibir Mubarat.
						
						Following the Firiik loss in the First Lashnar-Firiik War, many of the minor Mubarats found themselves placed in an awkward position. The Confederation, while defeated, was still a strong power. The remaining Mubarats - particularly the major Mubarats of Allogron, Kurfin, Durris, and Maidan - become closer and more intertwined. Together they demanded more payments and tributes from the minor Mubarats under their authority, citing the ineffectiveness of the minor Mubarats during the war against the upstart ex-slave empire and blaming the loss on them. The law was passed in Allogron following a unanimous declaration among the major Mubarats in the year 19 AC, upsetting several minor Mubarats. Additionally, following the Firiik loss there was a string of cancelled marriage plans between the major Mubarats and the minor. Several of the larger Mubarats - particularly Allogron and and Kurfin - felt it would be politically safer to increase the relations between themselves and other major Mubarats, and not the minor ones. This marks a continuing shift in thinking for the major Mubarats, which over the past century had increasingly seen the minor Mubarats as subjects rather than less-powerful equals. These movements angered many of the minor Mubarats and were the main reasons for the distancing of the Kilibir Mubarat away from the Confederation and towards new possibilities.
						
						The Kilibir Mubarat thus looked out to the north, where the disruption caused by the leaving of the Confederation opened up ample opportunities for a future for the Mubarat. Following Tinarlar's victory against Malietr, the Kilibir Mubar Firtik accepted Tinarlar's offer for diplomacy. Intrigued by the small Fistin's rapid gains and Tinarlar's claim to Firiik ancestry, Firtik saw in Tinarlar an opportunity for a politically safe partnership of sorts. It was Kilibir's support which allowed Tinarlar to rebound from his victory at Malietr and continue east against the Deggestin. 
					
						Tinarlar's strength ended up being in his effective use of diplomacy and charm over his military strength. By his death in 54 AC, he had conquered the northern bank of the Mubarat River (no small feat on its own) and established strong relationships between not only his subjects but also his neighbors. Tinar's relationship with Kilibir would eventually grow into one of marriage, which would finally turn the Tinar Fistin into a Mubarat.
\end{multicols}
