	The Bulkhai Sahinate was the ruling body of the Bulkhan Peninsula - and later, the Tokkan Plains, Rokali highlands, and Il'Vagur Peninsula - between the years 196 and 717 AC. It is one of the longest lasting continuous governments in history and the longest one in southern Ma'at. The Sahinate was the last government which held a unified peninsula until the de-colonization period following the Millenium War and the establishment of the Bulkhai Republic. Although the capital Tinar was lost in 717, the Sahinate continued to survive under the rule of the Dragmar until 723, although this is not considered a part of the Bulkhai Sahinate but a successor.
	
	Following the Babuas' Coup against the Tinar Mubar Gezuar II, the Bulkhai Sahinate was established as a way for the religious elite of the peninsula to maintain an overbearing influence on the government, a status which they would maintain until Beliar's reforms in 363 AC. Between 196 and 363 AC, the Sahinate was ruled in a de-centralized fashion, with the Zoginates maintaining an increasing level of autonomy and the Zogs charged with ruling these regions were allowed increasing levels of power. The Babuas' influence on the Sahinate was greatly diminished following Enver's War (359 - 361 AC) and Beliar's reforms of 363, which granted the Sahin increased authority, expanded the number of ruling bodies was three to five, and worked to centralize authority within the capital while creating a meritocratic system of advancement. The reforms eventually allowed Bulkhai to spread far beyond the peninsula, gaining influence and territories in the Tokkan Plains, lower Rokali Highlands, and southwestern Kriovuh coastline.
	
	The expansion of the Bulkhai Sahinate came at a cost of stability, particularly in the high government. An increasing amount of internal squabbling saw the Zoginates and the Babuas openly in conflict for several decades, attempting to wrest power both from one another and from the Sahin himself. However, this expansion also allowed for Bulkhai to expand its horizons and establish more direct contact with states outside of the Bulkhan Ring. Voyages under Bulkhan explorers, in particular Barkabir Aimat, would lead to the exploration of southern Vinayaka, north Ma'at, and north Enki. 
	
	The late 400s saw a decline in Bulkhan "extravagances". Following the Battle of Altivo (472 AC), Bulkhai would not capture any new territory until after Ximal's ascension to the position of Sahin in 521 AC. Between 472 and 521, Bulkhai saw a substantial decrease in its standing army, a loss of several territories in the Mai-Mai coastline and the Rokali Highlands, the Sahinate navy to fall into disrepair, and an increasing dependence on slavery in Valona and Girok (despite the practice being illegal against other Dregunist worshippers). In particular, the rise of slavery coincided with several tax laws and an updated census which resulted in increased taxes for those with more workers under their command. As a result, raids from Valona and Girok into the Rokali lands were quite frequent, where both Dregunist and non-Dregunist worshippers would be taken and enslaved without being listed on the census.
	
	Following the Scribe's Revolt (520 AC) and Ximal's accession to position of Sahin (521 - 566 AC), Bulkhai saw a massive gain in territory within the Tokkan Ring. Ximal oversaw the addition of the entirety of the Il'Vagur Peninsula, Mai-Mai Plateau, and the Tokkan Angxaos. The Sahinate maintained very loose control over its new territories, preferring to allow much of its newly conquered regions a great amount of autonomy as long as taxes were paid and Bulkhai was recognized as the Chosen of the Dregunist Faith. Bulkhai's Golden Age continued with its expansion into the Rokali Highlands between 578-596 AC, ending following Il'Poal's Rebellion (617 - 632 AC). 
	
	The 7th century saw the expansion of Pan Gu and Ullr trade throughout the world. Bulkhai, while not maintaining a completely isolationist stance, Bulkhai allowed foreign powers to lease ports within its borders, with Glimmer being allowed the vast majority of these ports. This increase in trade between 633 and 680 helped to maintain a declining empire. Bulkhai's lifespan was extended until the eruption of Maliri in 717 AC, which saw the complete loss of the High Government of Bulkhai and began the Wars of Bulkhan Succession, which signaled the beginning of the Feuding Cities Era and the end of the Bulkhai Sahinate.
	
\subsection{History}	
	\subsubsection{Founding}
		\paragraph{Decline of Firiir}
		\paragraph{Rise of Tinar}
		\paragraph{Tinar Expansion, Conflicts with Lashnar}
		\paragraph{Subjugation of Lashnar, Skudar}
		\paragraph{Unification with Valona}
	\subsubsection{Early Period}
		\paragraph{The Babuas and the Zoginates}
		\paragraph{Foreign Interventions}
	\subsubsection{Enver's Storm, War}
		\paragraph{Enver's Storm}
		\paragraph{Enver's War}
		\paragraph{Beliar's Reforms and Aftermath}
	\subsubsection{Beliar-Reform Period}
		\paragraph{Consolidation of authority}
		\paragraph{Expansion and Hegemony}
		\paragraph{Stagnation}
	\subsubsection{Ximal-Reform Period}
		\paragraph{Ximal, The Great: Early Life}
		\paragraph{Scribe's Revolt}
		\paragraph{Reform and Consolidation of Power}
		\paragraph{Expansion into the Il'Vagur Peninsula}
		\paragraph{Subjugation of the Mai-Mai Coastline and Plateau}
		\paragraph{The Tokkan Dragon Wars}
		\paragraph{Bulkahn Golen Age}
	\subsubsection{Decline}
		\paragraph{Global Exchange and Bulkhai}
				