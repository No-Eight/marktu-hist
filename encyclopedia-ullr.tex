\begin{multicols}{2}
	\section{Id}\index{Id}
			The Kingdom of Id\index{Id!Kingdom of} is a large nationstate encompassing the majority of Northern Ullr, comprised of an expansive and complex hierarchy of provinces ruled by local lords, and assembled into a parliamentary monarchy. Its terrain is poor for farming though rich in mineral and natural wealth, as it is mountainous, heavily wooded, and cold. Some of the farthest northern Iddish Lords remain tribal chiefs, though most of Id has since modernized into a feudal, then imperial, and finally metropolitan society.
	\subsection{Iddish History}
			For much of its history, Id has seemed to be a monolithic entity, inscrutable to outside eye and violent beyond all restraint. The latter claims are exaggeration while the former is an outright fabrication. While Iddish history has been marked with violence due in part to the scarcity of their land and large systems of familial alliances, Id is in no way a singular entity beyond being coalesced under a single overarching government. From the beginning of their known history, Id has been fragmented, and it is only in the modern era that the nation has somewhat solidified. 
		\subsubsection{Early History}\index{Id!Early History}\par
			From when mankind first entered the Iddish realms, they were confronted with inhospitable terrain. The ground is rocky, and a large portion of the realm is densely forested, rugged, and generally difficult to grow crops in. Most importantly, the winters of Id are quite severe, tending towards blizzard and deep freeze. Early Iddish life would have been extremely difficult, and when the first cities were first flourishing the Iddiat were still hunting and gathering.\par
			While their lands were poor, the Iddiat became very proficient at archery, hunting, and survival. Taking refuge in caves, usually hot spring caves that Id is known for, the Early Iddiat managed to build their first tribal communities. These tribes were heavily patriarchal, with the leader of the tribe being the father and all sons held in semi-equal regard beneath him. In importance, the elder sons were more powerful than the younger, but all sons were above importance of any daughters or wives within the familial group.\par 
			Entrance into a tribe would be done through marriage and ritual adoption. When a marriage was being negotiated, both of the Iddiat Patriarchs would come to an agreement of how the two tribes would merge. Usually, this could result in either of the two fathers gaining a gift while the other absorbs the son, and any of that son's subservients, into his family, or the two tribes integrating into one, with the two patriarchs becoming sworn brothers. Through this, the first traces of the infamous Iddiat Houses\index{Id!Iddiat Houses}.\par
			Early Id was marked with endemic warfare due to scarcity of game and fruit. Tribes would frequently raid in search of food, either other Iddiat tribes or any traders taking advantage of the central Iddish Expanse as an easier route between East and West Ullr. In Iddiat society, anyone who was not related to a tribe was considered a fair target for that tribe's raids.\par 
			Late in the Iddish Early History, these tribes began to mark out their territories as areas near where their wintering caves. Cave art depicting territorial maps have been found in since abandoned Iddiat Wintering Caves, as well as some other spiritual art, depicting animistic gods and mountain deities to the far north. Drawings of stellar constellations, such as the Ice Bear, or the Sky Tree, have also been found in their early art. Iddiat society began to settle down, and Ancient Skadunism took root within their society. To describe this, notable Iddish historian Alexei Burgensen has said "The Early Iddiat did not believe in their superstitions. They feared them."\par
		\subsubsection{Formation of Mountain-hearths}\par
			Though the climate was inhospitable, agriculture eventually entered Id through Bialka, which had volcanoes and thus was fucking awesome at agriculture. With this, the Iddish began to construct settlements near the caves they once too refuge in, using the high mineral content from the hot springs to help aid growth and harvest, while taking shelter within those very caves during the winters as they have always done.
		\subsubsection{Early Feudal Period}\par
		\subsubsection{Entrance of Skadunism}\par
		\subsubsection{Demirbjorn Dynasty}\par
		\subsubsection{Crusader Era}\par
		\subsubsection{Conspiracy of Id}\par
		\subsubsection{First Anarchal Period}\par
		\subsubsection{Reformative Era}\par
		\subsubsection{Second Anarchal Period}\par
		\subsubsection{Yoriksen and Kupperstar Dynasties}\par
		\subsubsection{First Empire of Great Id}\par
		\subsubsection{Third Anarchal Period}\par
		\subsubsection{Second Empire of Great Id}\par
		\subsubsection{Iddish Renaissance}\par
		\subsubsection{Great Revolution}\par
		\subsubsection{Id during the Millennium War}\par
		\subsubsection{Postwar Id}\par
	\subsection{Law of Taboo}\index{Id!Taboo, Law of}
	\subsection{Ancient Iddish Mythology}\index{Id!Ancient Mythology}
	\subsection{Iddiat Houses}\index{Id!Iddiat Houses
	\subsection{Modern Iddish Society}\index{Id!Modern Society}
	\subsection{Iddish Culture}\index{Id!Culture}
	\subsection{Great Northern Road}\index{Trade!Ullr!Great Northern Road}
\end{multicols}