Tz'amtanak'al\index{Tz'amtanak'al} (Other Language Information) is a semi-autonomous City-State within Lu'um Utz-Xib\index{Lu'um Utz-Xib}. Tz'amtanak'al is one of the largest and most historically significant cities in Vinayaka, and is commonly considered to be among the oldest Tetk'in City-States. Even today, Tz'amtanak'al's population is majority Tetk'in.
	\subsection{Mythology}
		\index{Tz'amtanak'al!Mythology of}
	In Tetk'in Mythology, the Founding of Tz'amtanak'al occured shortly after the Tetk'in were delivered to the world by the heavens. According to myth, the Tetk'in were drawn to the future site of the city, a pristine marsh in the exact center of the world. The Tetk'in were drawn to this place due to their resonance with heaven, as it is said that above this spot the sun is directly overhead.\par
	After the Tetk'in had begun to live around the marsh, a great man measuring a full head taller than the tallest among them arrived, and it is said he broadcast the same warmth and radiance as the sun. Known as the Heavenly Ajaw [Name Needed], this man invited the Tetk'in to live atop the marsh, where they might be as close as possible to heaven. This Ajaw oversaw the foundation of the city atop that marsh, and also the creation of Tz'amtanak'al's Pyramid of the Sun\index{Tz'amtanak'al!Pyramid of the Sun}. \par 
	Once the city was complete, the Heavenly Ajaw pronounced it Tz'amtanak'al, meaning Throne in the Marsh, then departed from the world, instructing the people of his city to unify their brothers and drive out the barbarians living on the edges of the world.\par
	Tetk'in mythology glorifying Tz'amtanak'al is not unique to the city itself; many Tetk'in throughout West Vinayaka hold the city in high esteem or even reverence, and this myth is part of the standard mythology for much of the region. This mythologized view of Tz'amtanak'al has led to it becoming an object of pilgrimage and of desire, even to the modern day. This has led to multiple conflicts over control of the city, most notably the Siege of Tz'amtanak'al in the Tetk'in Civil War. It is only a slight exaggeration to call Tz'amtanak'al the holy city of Tetk'in folk religion.